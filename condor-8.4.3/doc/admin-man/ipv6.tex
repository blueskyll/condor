%%%%%%%%%%%%%%%%%%%%%%%%%%%%%%%%%%%%%%%%%%%%%%%%%%%%%%%%%%%%%%%%%%%%%%%%%%%
\subsection{\label{sec:ipv6}Running HTCondor on an IPv6 Network Stack}
%%%%%%%%%%%%%%%%%%%%%%%%%%%%%%%%%%%%%%%%%%%%%%%%%%%%%%%%%%%%%%%%%%%%%%%%%%
\index{IPv6|(}

By default, HTCondor uses only IPv4 networks.  To enable IPv6 networking,
set configuration variable \Macro{ENABLE\_IPV6} to \Expr{True}.
Do \emph{not} enable IP6 unless
the machine on which the daemon is running has an IPv6 address that the
other machines within the pool can reach.  
Otherwise, as HTCondor prefers to use IPv6 networks,
enabling IPv6 can cause connectivity problems.
In particular, it is unlikely IPv6 should be enabled if the IPv6 addresses
are loopback (::1) or link-local (fe80::).

To disable IPv4 networking, set configuration variable \Macro{ENABLE\_IPV4} 
to \Expr{False}.
HTCondor operates normally using only IPv6 addresses.

If both IPv4 and IPv6 networking are enabled, HTCondor runs in mixed mode.
In mixed mode, HTCondor daemons have at least two addresses that include
an IPv4 address and an IPv6 address.  
Generally, other daemons and the HTCondor client
tools will choose between these addresses based on which protocols are enabled
for them, preferring IPv6 when given the choice.

There are two important cases in which the preference will not be for an IPv6 address:
\begin{itemize}
\item{When given a literal IP address, HTCondor will use that IP address.  }
\item{When looking up a host name using DNS,
HTCondor will use the first address whose
protocol is enabled for the tool or daemon doing the look up.  }
\end{itemize}

In practice, this means that both an HTCondor pool's central manager
and any submit machines within a mixed mode pool must have both IPv4 and IPv6
addresses for both IPv4-only and IPv6-only \Condor{startd} daemons 
to function properly.

\subsubsection{IPv6 and Host-Based Security}

You may freely intermix IPv6 and IPv4 address literals.  You may also specify
IPv6 netmasks as a legal IPv6 address followed by a slash followed by the
number of bits in the mask; or as the prefix of a legal IPv6 address followed
by two colons followed by an asterisk.  The latter is entirely equivalent to the
former, except that it only allows you to (implicitly) specify mask bits
in groups of sixteen.  For example, \texttt{fe8f:1234::/60} and
\texttt{fe8f:1234::*} specify the same network mask.

The HTCondor security subsystem resolves names in the ALLOW and DENY
lists and uses all of the resulting IP addresses.  Thus, to allow or deny
IPv6 addresses, the names must have IPv6 DNS entries (AAAA records), or
\MacroNI{NO\_DNS} must be enabled.

\subsubsection{IPv6 Address Literals}

When you specify an IPv6 address and a port number simultaneously, you
must separate the IPv6 address from the port number by placing square
brackets around the address.  For instance:

\begin{verbatim}
COLLECTOR_HOST = [2607:f388:1086:0:21e:68ff:fe0f:6462]:5332
\end{verbatim}

If you do not (or may not) specify a port, do not use the square brackets.
For instance:

\begin{verbatim}
NETWORK_INTERFACE = 1234:5678::90ab
\end{verbatim}

\subsubsection{IPv6 without DNS}

When using the configuration variable \Macro{NO\_DNS},
IPv6 addresses are turned into host names by taking the IPv6 address,
changing colons to dashes, and appending \MacroUNI{DEFAULT\_DOMAIN\_NAME}.
So,
\begin{verbatim}
2607:f388:1086:0:21b:24ff:fedf:b520
\end{verbatim}
becomes
\begin{verbatim}
2607-f388-1086-0-21b-24ff-fedf-b520.example.com
\end{verbatim}
assuming
\begin{verbatim}
DEFAULT_DOMAIN_NAME=example.com
\end{verbatim}

\index{IPv6|)}
